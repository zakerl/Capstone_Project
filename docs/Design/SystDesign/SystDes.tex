\documentclass[12pt, titlepage]{article}

\usepackage{fullpage}
\usepackage[round]{natbib}
\usepackage{multirow}
\usepackage{booktabs}
\usepackage{tabularx}
\usepackage{graphicx}
\usepackage{float}
\usepackage{hyperref}
\usepackage{float}
\hypersetup{
    colorlinks,
    citecolor=blue,
    filecolor=black,
    linkcolor=red,
    urlcolor=blue
}

\input{../../Comments}
%% Common Parts

\newcommand{\progname}{Mechatronics Engineering} % PUT YOUR PROGRAM NAME HERE
\newcommand{\authname}{Team \#1, Back End Developers
\\ Jessica Bae
\\ Oliver Foote
\\ Jonathan Hai
\\ Anish Rangarajan
\\ Nish Shah
\\ Labeeb Zaker} % AUTHOR NAMES                  

\usepackage{hyperref}
    \hypersetup{colorlinks=true, linkcolor=blue, citecolor=blue, filecolor=blue,
                urlcolor=blue, unicode=false}
    \urlstyle{same}

\newcounter{acnum}
\newcommand{\actheacnum}{AC\theacnum}
\newcommand{\acref}[1]{AC\ref{#1}}

\newcounter{ucnum}
\newcommand{\uctheucnum}{UC\theucnum}
\newcommand{\uref}[1]{UC\ref{#1}}

\newcounter{mnum}
\newcommand{\mthemnum}{M\themnum}
\newcommand{\mref}[1]{M\ref{#1}}

\begin{document}

\title{System Design for \progname{}} 
\author{\authname}
\date{\today}

\maketitle

\pagenumbering{roman}

\section{Revision History}

\begin{tabularx}{\textwidth}{p{3cm}p{2cm}X}
\toprule {\bf Date} & {\bf Version} & {\bf Notes}\\
\midrule
18 January 2023 & 1.0 & Revision 0 for System Design\\
\bottomrule
\end{tabularx}

\newpage

\section{Reference Material}

This section records information for easy reference.

\subsection{Abbreviations and Acronyms}
Refer to \href{https://github.com/zakerl/Capstone_Project/blob/main/docs/SRS/SRS.pdf}{SRS} for a comprehensive list of abbreviations and acronyms.\\

\renewcommand{\arraystretch}{1.2}
\begin{tabular}{l l} 
  \toprule		
  \textbf{symbol} & \textbf{description}\\
  \midrule 
	Lipo battery & Lithium polymer battery\\
	PCB & Printed circuit board\\
	TFT LCD Display & Thin film transistor liquid crystal display\\
	FSM & Finite state machine\\
	UI & User interface\\
	CAD & Computer aided design for 3D models\\
	SPI & Serial peripheral interface\\

  \bottomrule
\end{tabular}\\


\newpage

\tableofcontents

\newpage

\listoftables

\listoffigures

\newpage

\pagenumbering{arabic}

\section{Introduction}

This document provides a detailed description of the system design for the EMAnator; the system currently under development by the Back End Developers which aims to assist researchers in performing Ecological Momentary Assessment for older adults. The goal of this design is to construct a system which fulfills all the requirements specified in the \href{https://github.com/zakerl/Capstone_Project/blob/desDoc_Labeeb/docs/SRS/SRS.pdf}{System Requirements Specification}, and that meets the needs of EMA researchers.\\

This document provides the big-picture goals of the system, an overview of the project, a comprehensive list of system variables, and details regarding the user interfaces of the system. It also lists the various hardware and electrical components involved in the design of the system. Finally, it includes a high-level timeline for the development of the system. \\

The design presented in this document is the result of collaboration between the Dr. Luciana Macedo of the School of Rehabilitative Sciences and the Back End Developers development team. We have discussed the project requirements and identified the best approach to meet them. This is covered in detail in the document titled \href{https://github.com/zakerl/Capstone_Project/blob/desDoc_Labeeb/docs/ProblemStatementAndGoals/Team1_ProblemStatement\%20\%26\%20Goals.pdf}{Problem Statement and Goals}.\\

The Back End Developers hope this document serves as a useful guide for anyone involved in the development or deployment of the system. \\


\section{Purpose}

In general, engineering design documentation is a set of documents that outline the detailed specifications for an engineering project. It describes the design of the project, including its requirements, the materials to be used, the processes involved, the safety and environmental considerations, and the estimated costs. The purpose of engineering design documentation is to provide a comprehensive record of the project that can be consulted by engineers, oversight bodies, and other stakeholders throughout the project’s life cycle.\\

This documentation includes technical drawings, process diagrams, system and component specifications, and relevant schematics and images. It also provides a basis for quality control and assurance, as well as a way to track progress and identify potential areas of improvement. It is an essential part of any engineering project, as it ensures that all stakeholders have a clear understanding of the project and the necessary steps for its successful completion.\\

This project currently has the current pieces of design documentation available:

\begin{itemize}
	\item \href{https://github.com/zakerl/Capstone_Project/blob/main/docs/ProblemStatementAndGoals/Team1_ProblemStatement\%20\%26\%20Goals.pdf}{Problem Statment and Goals}\\
	\item \href{https://github.com/zakerl/Capstone_Project/blob/main/docs/DevelopmentPlan/DevelopmentPlan.pdf}{Development Plan}\\
	\item \href{https://github.com/zakerl/Capstone_Project/blob/main/docs/SRS/SRS.pdf}{System Requirements Specification}\\
	\item \href{https://github.com/zakerl/Capstone_Project/blob/main/docs/HazardAnalysis/HazardAnalysis.pdf}{Hazard Analysis}\\
	\item \href{https://github.com/zakerl/Capstone_Project/blob/main/docs/VnVPlan/VnVPlan.pdf}{VnV Plan}\\
	\item \href{https://github.com/zakerl/Capstone_Project/blob/main/docs/Design/SoftArchitecture/MG.pdf}{Module Guide}\\
	\item \href{https://github.com/zakerl/Capstone_Project/blob/main/docs/Design/SoftDetailedDes/MIS.pdf}{Module Interface Specification}\\
	\item System Design Document\\
\end{itemize}

\section{Scope}

\wss{Include a figure that show the System Context (showing the boundary between
your system and the environment around it.)}

\section{Project Overview}

\subsection{Normal Behaviour}

Refer to \href{https://github.com/zakerl/Capstone_Project/blob/main/docs/SRS/SRS.pdf}{SRS} Section 12.

\subsection{Undesired Event Handling}

Refer to \href{https://github.com/zakerl/Capstone_Project/blob/main/docs/SRS/SRS.pdf}{SRS} Section 13.

\subsection{Component Diagram}

\subsection{Connection Between Requirements and Design} \label{SecConnection}

\wss{The intention of this section is to document decisions that are made
  ``between'' the requirements and the design.  To satisfy some requirements,
  design decisions need to be made.  Rather than make these decisions implicit,
  they are explicitly recorded here.  For instance, if a program has security
  requirements, a specific design decision may be made to satisfy those
  requirements with a password.}

\section{System Variables}

\subsection{Monitored Variables}

Refer to \href{https://github.com/zakerl/Capstone_Project/blob/main/docs/SRS/SRS.pdf}{SRS} Section 5.1.2.
\subsection{Controlled Variables}

Refer to \href{https://github.com/zakerl/Capstone_Project/blob/main/docs/SRS/SRS.pdf}{SRS} Section 5.1.3.

\subsection{Constants Variables}

Refer to \href{https://github.com/zakerl/Capstone_Project/blob/main/docs/SRS/SRS.pdf}{SRS} Section 1.4.
\section{User Interfaces}


\subsection{Hardware User Interface}

The device is worn by a participant on the wrist for measuring activity and generating  prompts. The following items will be shown on the display of the activity tracker:
\begin{table}[H]
	\begin{tabularx}{1.05\textwidth} { 
		  | >{\centering\arraybackslash}X 
		  | >{\centering\arraybackslash}X 
		  | >{\centering\arraybackslash}X 
		  | >{\centering\arraybackslash}X | }
		 \hline
		 \textbf{Description} & \textbf{Behaviour of TFT Display} \\
		 \hline
		Power up of activity tracker. & Displays Back End Developers on startup.\\
		\hline
		 Default behaviour, no activity tracked.  & Displays date and time.\\
		 \hline
		   Activity tracked. & Prompt generated on screen, for example: Are you in pain?\\
		\hline 
		Answering prompts using touch sensor (bezel). & Toggle between different 				options on screen. For example: (Yes/No).\\
		\hline
	\end{tabularx}
\caption{\label{Hardware User Interface}Components of Hardware UI}  
\end{table}

\begin{figure}[H]
	\begin{center}
		 \includegraphics[width=1\textwidth]{HardwareUI_FSM}
		\caption{FSM for Hardware UI}
		\label{HardwareUI_FSM} 
	\end{center}
\end{figure}

\begin{figure}[H]
	\begin{center}
		 \includegraphics[width=0.5\textwidth]{BEDDisplay}
		\caption{Display on activity tracker at startup}
		\label{BEDDisplay} 
	\end{center}
\end{figure}

\begin{figure}[H]
	\begin{center}
		 \includegraphics[width=0.5\textwidth]{DisplayTime}
		\caption{Display of Date/time on activity tracker.}
		\label{DisplayTime} 
	\end{center}
\end{figure}

\subsection{Software User Interface}

The Software User Interface will be used by the Researcher for configuring the activity tracker. The interface will be on the Host Computer and will be able to store participant data, create new data and view records using encryption. The interface will also have authentication, and only the Researcher will be able to log in. The following features are available on the Software User Interface:

\begin{table}[H]
	\begin{tabularx}{1.05\textwidth} { 
		  | >{\centering\arraybackslash}X 
		  | >{\centering\arraybackslash}X 
		  | >{\centering\arraybackslash}X 
		  | >{\centering\arraybackslash}X | }
		 \hline
		 \textbf{Options on UI} & \textbf{Description}\\
		 \hline
		Main window & Main menu that leads to different windows when clicked.\\
		\hline
		 Connect to tracker  & Connects to SD card for device and shows status of connection.\\
		 \hline
		   Create Records Window & Creates new record for particpant and stores it in a database. A record can only be created if the correct username and password is provided. \\
		\hline 
		Records Window & Participant records can be viewed in a tabular format and can be searched/filtered.\\
		\hline
		Data View Window & Data stored on SD card can be viewed and filtered. Data can also be plotted using Graph button. For example: Heart Rate vs Time.\\
		\hline
	\end{tabularx}
\caption{\label{Software User Interface}Components of Software UI}  
\end{table}

Below is an example of the Software User Interface for the Main window.
\begin{figure}[H]
	\begin{center}
		 \includegraphics[width=0.6\textwidth]{MainWindow}
		\caption{Main Window for Software UI}
		\label{MainWindow} 
	\end{center}
\end{figure}

For more examples of the Software User Unterface, refer to Appendix \ref{Software_UI}.

\begin{figure}[H]
	\begin{center}
		 \includegraphics[width=1\textwidth]{SoftwareUI_FSM}
		\caption{FSM for Software UI}
		\label{SoftwareUI_FSM} 
	\end{center}
\end{figure}

\section{Design of Hardware}

\begin{figure}[H]
	\begin{center}
		 \includegraphics[width=0.5\textwidth]{DisplayCase}
		\caption{TFT Display with custom 3D printed case}
		\label{DisplayCase} 
	\end{center}
\end{figure}


The touch sensor shown below will be used as a bezel to navigate options on the activity tracker. The wires connected to each segment send a corresponding signal which can be used to navigate options and select a response.

\begin{figure}[H]
	\begin{center}
		 \includegraphics[width=0.6\textwidth]{TouchSensor}
		\caption{Circuit of custom built reactive touch sensor}
		\label{TouchSensor} 
	\end{center}
\end{figure}

\begin{figure}[H]
	\begin{center}
		 \includegraphics[width=0.5\textwidth]{WatchCAD}
		\caption{CAD Assembly of Touch Bezel, Display case and TFT Display}
		\label{WatchCAD} 
	\end{center}
\end{figure}

\begin{figure}[H]
	\begin{center}
		 \includegraphics[width=0.5\textwidth]{PCBTOP}
		\caption{Top view of Custom PCB}
		\label{PCBTOP} 
	\end{center}
\end{figure}

\begin{figure}[H]
	\begin{center}
		 \includegraphics[width=0.5\textwidth]{PCBBOTTOM}
		\caption{Bottom view of Custom PCB (Lipo battery holder and SD card slot)}
		\label{PCBBOTTOM} 
	\end{center}
\end{figure}

Refer to Appendix \ref{MechHardware} for individual CAD designs.\\

The following list of items will be custom designed and fabricated:
\begin{itemize}
\item{Custom PCB designed using Easy EDA and fabricated using JLCPCB.}
\item{3D printed casing for TFT Display/activity tracker.}
\item{Touch sensor (bezel) to navigate activity tracker.}
\end{itemize}

Refer to table \ref{DesignHardware} in Appendix for detailed information on high-level hardware components used.

\section{Design of Electrical Components}

The schematic/circuit diagrams shown below are used to generate the PCB layout. It consists of the following modules:
\begin{itemize}
\item Seeduino Micro-controller
\item Touch sensor module
\item IO Expander
\item RTC
\item MPU 6050
\item SD card module
\item Lipo module
\end{itemize}

\begin{figure}[H]
	\begin{center}
		 \includegraphics[width=1\textwidth]{Schematic}
		\caption{Schematic for PCB}
		\label{Schematic} 
	\end{center}
\end{figure}

The custom designed PCB is shown below.

\begin{figure}[H]
	\begin{center}
		 \includegraphics[width=0.75\textwidth]{Layout}
		\caption{Layout of PCB design}
		\label{Layout} 
	\end{center}
\end{figure}

Refer to table \ref{DesignElectrical} in Appendix for detailed information on electrical components (module-wise) used in the PCB design.

\section{Design of Communication Protocols}

There are 2 standard communication protocols that are used in the project, I2C and SPI.
The following modules use I2C communication protocol:
\begin{itemize}
\item Code for RTC using I2C communication.
\item TFT display driver.
\item MPU 6050 driver.
\end{itemize}

The IO expander module uses SPI communication protocol.

\section{Timeline}

The project timeline shown below as per important tasks/modules are required for the implementation of the Revision 0 demonstration:

\begin{table}[H]
	\begin{tabularx}{1.05\textwidth} { 
		  | >{\centering\arraybackslash}X 
		  | >{\centering\arraybackslash}X 
		  | >{\centering\arraybackslash}X 
		  | >{\centering\arraybackslash}X | }
		 \hline
		 \textbf{Timeline} & \textbf{Project Task} & \textbf{Team member(s) responsible and duties} & \textbf{Description}\\
		 \hline
		Dec 2022 - Jan 2023 & Software UI and functionality & Labeeb Zaker (backend), Nish Shah (frontend) & User friendly UI implemented with authentication and encryption working. \\
		\hline
		Jan 2022 & Database Management and writing to SD card & Nish Shah (DBMS), Jessica Bae (Communication protocols) & Implement database with Software UI to read, store and write data. Storing data read from sensor in proper format on SD card.\\
		\hline
		Dec 2022- Jan 2023 & PCB schematic design & Anish Rangarajan (General schematics), Jessica Bae (SDCARD module), Jonathan Hai (Lipo module) & Implement schematics for different modules in PCB. \\
		\hline
		Dec 2022 - Jan 2023 & PCB layout design & Anish Rangarajan  & Implement layout of PCB with proper routing.\\
		\hline
		Nov 2022 & Display drivers/CAD models & Labeeb Zaker (Drivers), Anish Rangarajan (CAD) & Display drivers for LCD display and CAD models for device.\\
		\hline
		Dec 2022 -Feb 2023 & Integration of all sensor modules & Jonathan Hai, Oliver Foote & Integrate all sensors and systems and create test benches to check correct 			functionality.\\
		\hline
			Jan-March 2023 & Fabrication of PCB and Testing  & Jessica Bae, Oliver Foote & Fabricating PCB and System-wide testing on PCB based on \href{https://github.com/zakerl/Capstone_Project/blob/main/docs/VnVPlan/VnVPlan.pdf}{VnVPlan} \\
		\hline
		Continuous development till March 2023 & Documentation & All team members & Various project documentations. \\
		\hline
	\end{tabularx}
\caption{\label{Timeline}Timeline of project}  
\end{table}


% \bibliographystyle {plainnat}
% \bibliography{../../../refs/References}

\newpage{}

\appendix

\section{Software Interface}
\label{Software_UI}
\begin{figure}[H]
	\begin{center}
		 \includegraphics[width=0.8\textwidth]{Config}
		\caption{Configuration Window}
		\label{Config} 
	\end{center}
\end{figure}

\begin{figure}[H]
	\begin{center}
		 \includegraphics[width=0.5\textwidth]{CreateRecord}
		\caption{Create Record Window}
		\label{CreateRecord} 
	\end{center}
\end{figure}

\begin{figure}[H]
	\begin{center}
		 \includegraphics[width=1\textwidth]{Record}
		\caption{Record Window}
		\label{Record} 
	\end{center}
\end{figure}

\begin{figure}[H]
	\begin{center}
		 \includegraphics[width=0.5\textwidth]{login}
		\caption{Login Dialogue box}
		\label{Login} 
	\end{center}
\end{figure}

\begin{figure}[H]
	\begin{center}
		 \includegraphics[width=1\textwidth]{DataView}
		\caption{Data View Window}
		\label{DataView} 
	\end{center}
\end{figure}

\begin{figure}[H]
	\begin{center}
		 \includegraphics[width=0.8\textwidth]{Graph}
		\caption{Generated Graph for Heart Rate vs Time}
		\label{Graph} 
	\end{center}
\end{figure}
\section{Mechanical Hardware}
\label{MechHardware}
\begin{figure}[H]
	\begin{center}
		 \includegraphics[width=0.5\textwidth]{BezelCAD}
		\caption{CAD model of touch sensor (bezel)}
		\label{BezelCAD} 
	\end{center}
\end{figure}

\begin{figure}[H]
	\begin{center}
		 \includegraphics[width=0.5\textwidth]{DisplayCaseCAD}
		\caption{CAD model of case for TFT Display}
		\label{DisplayCaseCAD} 
	\end{center}
\end{figure}

\begin{figure}[H]
	\begin{center}
		 \includegraphics[width=0.5\textwidth]{WatchCAD2}
		\caption{CAD model of watch assembled}
		\label{WatchCAD2} 
	\end{center}
\end{figure}

\begin{table}[H]
	\begin{tabularx}{1.05\textwidth} { 
		  | >{\centering\arraybackslash}X 
		  | >{\centering\arraybackslash}X 
		  | >{\centering\arraybackslash}X 
		  | >{\centering\arraybackslash}X | }
		 \hline
		 \textbf{Hardware Component} & \textbf{Description}\\
		 \hline
		Custom PCB & Custom PCB designed to fit in activity tracker.  \\
		\hline 
		Touch sensor module & Custom touch sensor (bezel) used to interact with activity tracker.\\
		\hline
		Outer casing for TFT Display & Designed using Autodesk Inventor and 3D printed. \\
		\hline 
		AT42QT1070-SSUR & Off-shelf microchip for capacitive touch sensors, used in touch sensor schematic.\\
		\hline
		 MPU 6050 & Accelrometer/Gyroscope, off-shelf component.\\
		\hline
		 Seeeduino xiao samd21  & off-shelf microcontroller for activity tracker.\\
		 \hline
		   DS1307 RTC & Real time clock, off-shelf component. \\
		\hline
		Waveshare 1.28 LCD-192 Display & Off-shelf TFT display used in activity tracker. \\
		\hline 
		Pulse sensor & Plug and play Heart Rate sensor for Arduino (off-shelf component)\\
		\hline 
		Li-Po Battery & Generic off-shelf lipo battery used for smart watches.  \\
		\hline 
		TP4056 Li-Ion BMS & Off-shelf constant-current/constant-voltage linear charge for battery, used in Lipo schematic.\\
		\hline
		DW01A-G & Off-shelf battery protection IC  to protect battery from overcharge, used in Lipo schematic.\\
		\hline
		USB Type-B charger & Generic off-shelf usb to Type-B charger to charge device. \\
		\hline 
		MicroSD card & Standard off-shelf SD card. \\
		\hline
		SDCARD connector 473521001& MicroSD connector (off-shelf component) \\
		\hline
		IO Expander SX1509BIULTRT & off-shelf IO Expander used in PCB.  \\
		\hline 
		Watch straps & Generic watch-straps for strapping device onto the wrist. \\
		\hline 
	\end{tabularx}
\caption{\label{DesignHardware}Components of Hardware Design}  
\end{table}

\section{Electrical Components}
\begin{table}[H]
	\begin{tabularx}{1.05\textwidth} { 
		  | >{\centering\arraybackslash}X 
		  | >{\centering\arraybackslash}X 
		  | >{\centering\arraybackslash}X 
		  | >{\centering\arraybackslash}X | }
		 \hline
		 \textbf{Module from schematic} & \textbf{Electrical component}\\
		\hline
		 MPU 6050  & \begin{itemize}
						\item{Capacitors (2nF, 10nF, 100nF)}
						\item{Resistors (4.7k ohms)}
					\end{itemize}\\
		\hline
		 Touch sensor  & \begin{itemize}
						\item{Capacitors (0.1uF)}
						\item{Resistors (470 ohms, 10k ohms)}
					\end{itemize}\\
		 \hline
		 IO Expander   & \begin{itemize}
						\item{Capacitors (100nF)}
						\item{Resistors (10k ohms)}
					\end{itemize}\\
		 \hline
		 RTC  & \begin{itemize}
						\item{Capacitors (15pF, 100nF)}
						\item{Resistors (4.7k ohms)}
					\end{itemize}\\
		 \hline
		 SD CARD Module   & \begin{itemize}
						\item{Capacitors (100nF)}
						\item{Resistors (10k ohms)}
					\end{itemize}\\
		 \hline
		 Lipo Module  & \begin{itemize}
						\item{Capacitors (100nF, 10uF)}
						\item{Resistors (100 ohms, 1k ohms, 1.2k ohms)}
						\item{FS8205 MOSFET}
					\end{itemize}\\
		 \hline
		 LEDs  & Various LEDs in modules used to identify status of operation.\\
		 \hline
	\end{tabularx}
\caption{\label{DesignElectrical}Electrical Components of modules from PCB schematic}  
\end{table}

\section{Communication Protocols}

\section{Reflection}

The information in this section will be used to evaluate the team members on the
graduate attribute of Problem Analysis and Design.  Please answer the following questions:

\begin{enumerate}
  \item What are the limitations of your solution?  Put another way, given
  unlimited resources, what could you do to make the project better? (LO\_ProbSolutions)
  \item Give a brief overview of other design solutions you considered.  What
  are the benefits and tradeoffs of those other designs compared with the chosen
  design?  From all the potential options, why did you select documented design?
  (LO\_Explores)
\end{enumerate}


\end{document}