\documentclass[12pt]{article}

\usepackage{multirow}
\usepackage{booktabs}
\usepackage{tabularx}
\usepackage{longtable}
\usepackage{graphicx}
\usepackage[table]{xcolor}
\usepackage[letterpaper, portrait, margin=1in]{geometry}
\usepackage{helvet}
\usepackage{float}
\renewcommand{\familydefault}{\sfdefault}

\title{Development Plan\\\progname}

\author{\authname}
\setlength\parindent{0pt}
\date{}

\input{../Comments}
%% Common Parts

\newcommand{\progname}{Mechatronics Engineering} % PUT YOUR PROGRAM NAME HERE
\newcommand{\authname}{Team \#1, Back End Developers
\\ Jessica Bae
\\ Oliver Foote
\\ Jonathan Hai
\\ Anish Rangarajan
\\ Nish Shah
\\ Labeeb Zaker} % AUTHOR NAMES                  

\usepackage{hyperref}
    \hypersetup{colorlinks=true, linkcolor=blue, citecolor=blue, filecolor=blue,
                urlcolor=blue, unicode=false}
    \urlstyle{same}

\begin{document}

\maketitle
\newpage

\tableofcontents
\listoftables

\newpage

\begin{table}[hp]
\caption{Revision History} \label{TblRevisionHistory}
\begin{tabularx}{\textwidth}{llX}
\toprule
\textbf{Date} & \textbf{Developer(s)} & \textbf{Change}\\
\midrule
September 26 & Back End Developers & Initial documentation\\
\bottomrule
\end{tabularx}
\end{table}

\pagebreak

\section{Team Meeting Plan}
Weekly meetings are to be held every Saturday at 8:00PM, (with an alternative option of having the meeting moved to Thursday at 7:00PM should any difficulties arise in hosting a meeting at the normal time) and additional meetings are to be planned as needed. Any meetings requiring the attendance of the team's supervisor will have to be planned based on Dr. Macedo's availability. In cases where some team members are absent, each absent member is expected to bring forward their discussion topics prior to the scheduled meeting time, and review the meeting notes posted in the Discord server. Note-taker is expected to write down meeting notes for each meeting, regardless of the number of participating members.\\

\pagebreak

\section{Team Communication Plan}
All primary communication is to be done through the team's Discord server, including all meetings. Secondary communication will be through the Facebook group chat only if necessary.\\

The following is a screen shot of this team's Discord server:\\
\begin{center}
    \includegraphics{DiscordServer}
\end{center}

\section{Team Member Roles}

\begin{table}[H]
\noindent
\begin{tabular}{ | m {3.3cm} | m {3cm} | m{9cm} | }
  \hline
 \cellcolor{black}\color{white}\textbf{Non-Technical Role} & \cellcolor{black}\color{white}\textbf{Name} & \cellcolor{black}\color{white}\textbf{Description}\\ [10pt]
  \hline
 Team Leader & Jonathan Hai \& Jessica Bae & Organize meetings, assign specific tasks to all team members, overlook deliverable deadlines, keep track of overall team project progress, communicate with the TA, the professor, and the supervisor \\
 \hline
 GitHub Leader & Labeeb Zaker \& Nish Shah & Review and approve all Git merge requests from the team.  \\  
 \hline
 Meeting Coordinator & Anish Rangarajan & Lead the conversation for team meetings. Responsible for preparing meeting agenda.\\
 \hline
 Note-Taker & Oliver Foote & Take notes during the meeting and post it in the meeting-notes section of Discord channel for record keeping purposes.\\
   \hline
\end{tabular}
\caption{\label{techRoles}Non-technical Roles and their Assignees}
\end{table}

\begin{table}[H]
\begin{tabular}{ | m {3.3cm} | m {3cm} | m{9cm} | }
  \hline
 \cellcolor{black}\color{white}\textbf{Technical Role} & \cellcolor{black}\color{white}\textbf{Name} & \cellcolor{black}\color{white}\textbf{Description}\\ [10pt]

   \hline
 Embedded \newline Developer & Jessica Bae & Responsible for designing and writing code to control the device in the system.\\
\hline
Frontend \newline Developer & Oliver Foote & Responsible for designing the user-interacting portion of the system.\\
\hline
 Backend \newline Developer & Jonathan Hai & Responsible for designing and writing code to allow database and applications to communicate.\\
\hline
CAD Designer & Anish \newline Rangarajan & Responsible for designing the physical form factor of the device.\\
\hline
Electromechanical Designer & Labeeb Zaker \& Nish Shah & Responsible for calculations and design related to electromechanical systems within the device.\\
\hline
\end{tabular}
\caption{\label{techRoles}Technical Roles and their Assignees}
\end{table}

The task of reviewing rubrics, and each subcomponents of the project will not be assigned to any specific members. These tasks are to be allocated to different members each time as and when needed.\\

\pagebreak

\section{Workflow Plan}
\begin{enumerate}
\item Move current task into the To-Do Section of the Trello Board. Assign relevant labels through the Trello Card (Hardware, Software, Features, Bugs, etc.).
\item Make branch for current task on Github (Branch named according to: "Trello Ticket \# - Member Name").
\item Once task has been completed, move task into Testing Section of the Trello Board.
\item If testing is successful, send pull request and move task into the Code Review Section of the Trello Board.
\item Update code with any comments from other developers.
\item Once the pull request has been approved, move the task into the Finished Section of the Trello Board.
\item Inform team that the branch has been merged into the main branch.
\item Delete your extant branch.
\end{enumerate}



\section{Proof of Concept Demonstration Plan}

The risks for the success of the project lie primarily within the software that is designed. The hardware component in which data is collected will be designed with a relatively simplistic plug-and-play approach with the goal of feeding data to the software component of the project. Several off-the-shelf sensors will be used to reduce the development overhead and the many possible problems that developing a complex electromechanic system can involve. However, a proof-of-concept demonstration is still necessary regarding the ergonomics of the project. Should the hardware take on the form of wearable technology (as is currently planned), a demonstration is necessary to show that the system is possible to be worn comfortably, safely, and unintrusively during the user's daily activities.\\

The software system will be the lynchpin of the project. Its goal will be to take the various events detected by the hardware system that are relevant to Ecological Momentary Assessment (EMA) and handle them accordingly. This will involve logging timestamps, locations, environmental factors, and other momentary information, and prompting the user to answer self-survey questions relevant to their current situation. The software system will then process the answers to these questions and the momentary data. Finally it will send off the data to the team of physicians handling the user's EMA at the end of the monitoring period.\\

Other than the ergonomics, if the hardware component of the system is incapable of sending data and handling output from the software, continuing with the project involves finding a new method of data collection and transfer. If the software component of the system is incapable of responding to data inputs from the hardware, processing said data, and returning meaningful results, then the goal of the entire project is rendered moot.\\

The goal for the proof of concept demo will be to demonstrate a functional EMA-enabled software system that can respond to events relevant to EMA, and respond accordingly. This software system will be run locally on a team member's computer. It will be capable of:\\

\begin{itemize}
\item Accepting inputs in the form of momentary data from events that are triggered as a simulation of real EMA events (such as limps, falls, strange movement patterns, etc.)
\item Prompting the user to answer relevant self-survey questions.
\item Displaying information and intaking information from the user simplistically according to common HCI guidelines.
\item Processing the inbound EMA and survey data and producing results meaningful to the physicians responsible for the user's EMA.
\item Producing graphical representations of said EMA data.
\item Sending the processed data and representations to the physicians responsible for the user's EMA.
\end{itemize}

Regarding the hardware component of the project, it will be:\\

\begin{itemize}
\item Useable/wearable in a safe, comfortable, and non-intrusive manner optimized for human well-being and overall system performance.
\item Placed in a position to collect data relevant to EMA display data regarding EMA related activities.
\end{itemize}

\section{Technology}
The following tools shall be used for development in software, embedded systems as well as any support platforms in the system:

\begin{enumerate}
\item Programming languages:
\begin{itemize}
\item Python will be used to generate computer vision code along with several graphical and UI elements. If necessary, Python will also be used for Machine Learning integration.
\item C will be used to program the embedded system (micro-controllers) for efficiency and memory constraints.
\item SQL will be used for DBMS and storing data for activity-based tracking.
\end{itemize}
\item Platforms for development:
\begin{itemize}
\item VS code will be used for code-based development.
\item Arduino IDE may be used to code for several existing sensor libraries (gyro meters, Bluetooth modules, Wi-fi modules, etc).
\item STM32Cube IDE for STM32 development. 
\item Autodesk Inventor may be used for creating models for the device and finishing product.
\item MySQL will be used as the platform for interacting with the Database.\\
\end{itemize}

\item Version control platforms:
\begin{itemize}
\item GitHub will be used for version control for development purposes and general group activity tracking.
\item GitLab will be used for version control (collaborative) with McMaster University.
\end{itemize}
\item Document generation:
\begin{itemize}
\item Latex will be used for generating documentation. TeX distributions that will be used amongst the collaborators are  Texmaker, TeXworks, MikteX and VScode Latex extension.
\end{itemize}
\item Specific plans for Continuous Integration (CI)
\begin{itemize}
\item CI will be used to block merge requests until all tests have been passed and approval has been given by atleast two members of the team. 
\item This will speed up development by reducing the overall bugs introduced into the project.
\end{itemize}
\item Measuring tools for code:
\begin{itemize}
\item Coverage.py for measuring code coverage in Python and effectiveness of testing.
\item Valgrind for efficient memory management.
\item Bullseye coverage for C/C++.
\end{itemize}
\item Libraries to use:
\begin{itemize}
\item OpenCV: Open Source computer vision library capable of performing image and video manipulation.
\item Numpy, pandas: Libraries with Python for number and matrix computation/manipulation.
\item HAL libraries: Should we pursue using an ARM architecture chip, the HAL libraries will allow for ease of coding.
\item PyQt5: Python UI library.
\item Several Arduino Sensor libraries to use for interacting with accelerators and other wireless modules.
\end{itemize}
\item Tools to use for project:
\begin{itemize}
\item 3D printer and slicer tools for rapid prototyping.
\item Soldering station and Glue Gun.
\item Wood working tools (Hand Saws and Power Sanding ) if necessary.
\end{itemize}
\end{enumerate}

\section{Limitations of Tools and Methods of Solving}

\begin{itemize}
\item Python: Memory usage may be inefficient due to limitations of hardware. \\ Solution: Use C or utilize static memory allocation.
\item Arduino IDE: Libraries may be specific to the Arduino Microcontrollers. \\ Solution: Write drivers for other microcontrollers.
\item CI/CD/Code Measuring Tools: Complex to set up, may require extra learning overhead to reach full functionality. \\ Solution: Assign members to investigate and document CI/CD methodology.
\item Libraries: Compatibility of libraries are version dependent and not necessarily backwards or forward compatible. \\ Solution: Run compatibility testing before using or changing libraries and keep updated documentation available.
\item 3D Printing: High time constraints, low availability, and print quality may be inconsistent. \\ Must take time constraints into consideration before integrating 3D Printing into the project.
\end{itemize}

\pagebreak

\section{Coding Standard}
\begin{enumerate}
\item Python:
\begin{itemize} 
\item Code development will closely follow the Google Python Style Guide as a list for do's and don'ts for Python programs.
\item For inline code documentation, Google style doc-strings will be followed for commenting.
\item Pylint will be used as a linter for code development. This linter can be used as an extension to VS code while developing and debugging code.
\end{itemize}
\item C: lint will be used as a linter when programming in C for development and debugging purposes.
\end{enumerate}

\section{Project Scheduling}

\begin{enumerate}
\item Master Project Schedule will be made with a list of all the deadlines, deliverables and Project Implementation tasks. Work will be done based on weekly sprints where tasks will be assigned to each member along with an estimated number of days needed to finish the task. This number is evaluated based on relative complexity of each task.\\

\item Imposed deadline for all tasks are \textbf{two} days before the due date of the deliverable. On this day, the team will conduct a review of all the work done by the members and provide any feedback. The last day before the due date is reserved for making any final changes and small revisions. Moreover, the team will check the completed deliverable with the marking rubric.\\

\item Every two weeks, a design overhaul will be conducted wherein the team will go through the previous parts of the project and update it with any changes made as a result of the current iteration of the project.\\
\end{enumerate}







\end{document}