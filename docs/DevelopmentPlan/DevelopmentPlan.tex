\documentclass{article}

\usepackage{booktabs}
\usepackage{tabularx}
\usepackage{graphicx}

\title{Development Plan\\\progname}

\author{\authname}
\setlength\parindent{0pt}
\date{}

\input{../Comments}
%% Common Parts

\newcommand{\progname}{Mechatronics Engineering} % PUT YOUR PROGRAM NAME HERE
\newcommand{\authname}{Team \#1, Back End Developers
\\ Jessica Bae
\\ Oliver Foote
\\ Jonathan Hai
\\ Anish Rangarajan
\\ Nish Shah
\\ Labeeb Zaker} % AUTHOR NAMES                  

\usepackage{hyperref}
    \hypersetup{colorlinks=true, linkcolor=blue, citecolor=blue, filecolor=blue,
                urlcolor=blue, unicode=false}
    \urlstyle{same}

\begin{document}

\begin{table}[hp]
\caption{Revision History} \label{TblRevisionHistory}
\begin{tabularx}{\textwidth}{llX}
\toprule
\textbf{Date} & \textbf{Developer(s)} & \textbf{Change}\\
\midrule
September 26 & N/A & Initial documentation\\
\bottomrule
\end{tabularx}
\end{table}

\newpage

\maketitle

\wss{Put your introductory blurb here.}

\section{Team Meeting Plan}
Weekly meetings to be held every Saturday at 8:00PM, and additional meetings to be planned as needed. Any meetings requiring the attendance of supervisor will have to be planned based on Dr. Macedo's availability each time. In cases of team members' absence, each absent team members are expected to bring forward their discussion topics prior to the scheduled meeting time, and review the meeting notes posted in the Discord server. Team leaders are expected to write down meeting notes for each meeting, regardless of the number of participating members.\\

\section{Team Communication Plan}
All main communication channel will be done through the Discord server, including all meetings. Secondary communication will be through Facebook group chat if necessary.\\

The following is a screen shot of this team's Discord server:\\
\begin{center}
    \includegraphics{DiscordServer}
\end{center}

\section{Team Member Roles}
The team will have 2 team leaders (co-leaders), Jonathan Hai and Jessica Bae will take this role. The responsibilities of team leaders are to organize meetings, record meeting notes, assign specific tasks to all team members, overlook deliverable deadlines, keep track of overall team project progress, and communicate with the TA, the professor, and the supervisor.\\

The team will also have a Github leader, responsible for managing all technical Github merge requests and approving them. Labeeb Zaker will take on this role.\\

The task of eviewing rubrics, and each subcomponents of the project will not be assigned to any specific member. These tasks are to be allocated to different members each time, to be decided as needed.\\

\section{Workflow Plan}
For any documentation, team members are not required to put in a merge request. For any technical commits, everyone is required to put in a merge request for the Github leader to review and approve. Trello will be used for project management purposes (non-technical tasks) and Github issue tracker will be used for technical tasks.\\

\section{Proof of Concept Demonstration Plan}

What is the main risk, or risks, for the success of your project?  What will you
demonstrate during your proof of concept demonstration to convince yourself that
you will be able to overcome this risk?

\section{Technology}
The following tools shall be used for development in software, embedded systems as well as any support platforms in the system:

\begin{enumerate}
\item Programming languages:
\begin{itemize}
\item Python will be used to generate computer vision code along with several graphical and UI elements. If necessary, Python will also be used for Machine Learning integration.
\item C will be used to program the embedded system (micro-controllers) for efficiency and memory constraints.
\item C\# will be used the team decides to use Unity to create parts of the product.
\item SQL will be used for DBMS and storing data for activity-based tracking.
\end{itemize}
\item Platforms for development:
\begin{itemize}
\item VS code will be used for code-based development.
\item Arduino IDE may be used to code for several existing sensor libraries (gyro meters, Bluetooth modules, Wi-fi modules, etc).
\item STM32Cube IDE for STM32 development. 
\item Autodesk Inventor may be used for creating models for the device and finishing product.
\item MySQL will be used as the platform for interacting with the Database.
\end{itemize}
\item Version control platforms:
\begin{itemize}
\item GitHub will be used for version control for development purposes and general group activity tracking.
\item GitLab will be used for version control (collaborative) with McMaster University.
\end{itemize}
\item Document generation:
\begin{itemize}
\item Latex will be used for generating documentation. TeX distributions that will be used amongst the collaborators are  Texmaker, TeXworks, MikteX and VScode Latex extension.
\end{itemize}
\item Specific plans for Continuous Integration (CI), or an explanation that CI
  is not being done
\item Measuring tools for code:
\begin{itemize}
\item Coverage.py for measuring code coverage in Python and effectiveness of testing.
\item Valgrind for efficient memory management.
\item Bullseye coverage for C/C++.
\end{itemize}
\item Libraries to use:
\begin{itemize}
\item OpenCV: Open Source computer vision library capable of performing image and video manipulation.
\item Numpy, pandas: Libraries with Python for number and matrix computation/manipulation.
\item HAL libraries: Should we pursue using an ARM architecture chip, the HAL libraries will allow for ease of coding.
\item tkinter: Default python UI library.
\item Several Arduino Sensor libraries to use for interacting with accelerators and other wireless modules.
\end{itemize}
\item Tools to use for project:
\begin{itemize}
\item 3D printer and slicer tools for rapid prototyping.
\item Soldering station and Glue Gun.
\item Wood working tools (Hand Saws and Power Sanding ) if necessary.
\end{itemize}
\end{enumerate}

\section{Coding Standard}
\begin{enumerate}
\item Python:
\begin{itemize} 
\item Code development will closely follow the Google Python Style Guide as a list for do's and don'ts for Python programs.
\item For inline code documentation, Google style doc-strings will be followed for commenting.
\item Pylint will be used as a linter for code development. This linter can be used as an extension to VS code while developing and debugging code.
\end{itemize}
\item C: lint will be used as a linter when programming in C for development and debugging purposes.
\end{enumerate}

\section{Project Scheduling}

\begin{enumerate}
\item Master Project Schedule will be made with a list of all the deadlines, deliverables and Project Implementation tasks. Work will be done based on weekly sprints where tasks will be assigned to each member along with an estimated number of days needed to finish the task. This number is evaluated based on relative complexity of each task.\\

\item Imposed deadline for all tasks are \textbf{two} days before the due date of the deliverable. On this day, the team will conduct a review of all the work done by the members and provide any feedback. The last day before the due date is reserved for making any final changes and small revisions. Moreover, the team will check the completed deliverable with the marking rubric.\\

\item Every two weeks, a design overhaul will be conducted wherein the team will go through the previous parts of the project and update it with any changes made as a result of the current iteration of the project.\\
\end{enumerate}







\end{document}