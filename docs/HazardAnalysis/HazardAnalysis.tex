\documentclass{article}

\usepackage{booktabs}
\usepackage{tabularx}
\usepackage{hyperref}
\usepackage{color, colortbl}
\usepackage{multirow,graphicx,array}
\usepackage{enumitem}
\definecolor{LightCyan}{rgb}{0.88,1,1}

\hypersetup{
    colorlinks=true,       % false: boxed links; true: colored links
    linkcolor=red,          % color of internal links (change box color with linkbordercolor)
    citecolor=green,        % color of links to bibliography
    filecolor=magenta,      % color of file links
    urlcolor=cyan           % color of external links
}

\title{Hazard Analysis\\\progname}

\author{\authname}

\date{}

%% Comments

\usepackage{color}

\newif\ifcomments\commentstrue %displays comments
%\newif\ifcomments\commentsfalse %so that comments do not display

\ifcomments
\newcommand{\authornote}[3]{\textcolor{#1}{[#3 ---#2]}}
\newcommand{\todo}[1]{\textcolor{red}{[TODO: #1]}}
\else
\newcommand{\authornote}[3]{}
\newcommand{\todo}[1]{}
\fi

\newcommand{\wss}[1]{\authornote{blue}{SS}{#1}} 
\newcommand{\plt}[1]{\authornote{magenta}{TPLT}{#1}} %For explanation of the template
\newcommand{\an}[1]{\authornote{cyan}{Author}{#1}}

%% Common Parts

\newcommand{\progname}{ProgName} % PUT YOUR PROGRAM NAME HERE
\newcommand{\authname}{Team \#, Team Name
\\ Student 1 name and macid
\\ Student 2 name and macid
\\ Student 3 name and macid
\\ Student 4 name and macid} % AUTHOR NAMES                  

\usepackage{hyperref}
    \hypersetup{colorlinks=true, linkcolor=blue, citecolor=blue, filecolor=blue,
                urlcolor=blue, unicode=false}
    \urlstyle{same}
                                


\begin{document}

\maketitle
\thispagestyle{empty}

~\newpage

\pagenumbering{roman}

\begin{table}[hp]
\caption{Revision History} \label{TblRevisionHistory}
\begin{tabularx}{\textwidth}{llX}
\toprule
\textbf{Date} & \textbf{Developer(s)} & \textbf{Change}\\
\midrule
Date1 & Name(s) & Description of changes\\
Date2 & Name(s) & Description of changes\\
... & ... & ...\\
\bottomrule
\end{tabularx}
\end{table}

~\newpage

\tableofcontents

~\newpage

\pagenumbering{arabic}

\wss{You are free to modify this template.}

\section{Introduction}

\wss{You can include your definition of what a hazard is here.}

\section{Scope and Purpose of Hazard Analysis}

\section{System Boundaries and Components}

\section{Critical Assumptions}

\wss{These assumptions that are made about the software or system.  You should
minimize the number of assumptions that remove potential hazards.  For instance,
you could assume a part will never fail, but it is generally better to include
this potential failure mode.}

\begin{itemize}
\item No wires will come loose during use.
\item Batteries are pugged in correctly (the positive and negative ends are aligned as intended).
\item All data are stored in the correct memory location.
\item All subsystems work as intended.
\item All off-the-shelf components work was intended.
\end{itemize}

\section{Failure Mode and Effect Analysis}

\begin{tabular}{|m{6em}|m{5em}|m{5em}|m{5em}|m{5em}|m{9em}|}
\rowcolor{LightCyan}
\textbf{Design Component} &\textbf{Failure Modes}  &\textbf{Causes of Failure}
&\textbf{Effects of Failure} & \textbf{Detection} & \textbf{Recommended Action}
\tabularnewline\hline
	Data Storage  &
	Data stored at wrong memory location  &
	\begin{minipage}[t]{\linewidth}
        	\begin{itemize}[nosep, wide=0pt, leftmargin=*, after=\strut]
			\item Incorrect software commands
			\item Memory space doesn't exist (invalid memory selected)
			\item Insufficient memory space
			\item Physical damage to hardware memory chip 
        	\end{itemize}
	\end{minipage} & 
	Lost and unsaved data &
	Set up error handler to check if each data point is successfully stored at the correct memory location each time &
	Replace faulty hardware or set up correct memory path 
	\tabularnewline\cline{1-6}

   	% \multirow{-8}{*}{\centering\rotatebox[origin=c]{90}{Heart Rate Detection}}
        Data Storage &
	Data Stored with incorrect type &
	\begin{minipage}[t]{\linewidth}
		\begin{itemize}[nosep, wide=0pt, leftmargin=*, after=\strut]
			\item Wrong data type used for storing data
		\end{itemize}
	\end{minipage} &
	Analysis program can't interpret data &
	Failed data analysis &
	Convert data to correct type
	\tabularnewline\cline{1-6}

	Device manager &
	Unable to establish connection &
	\begin{minipage}[t]{\linewidth}
		\begin{itemize}[nosep, wide=0pt, leftmargin=*, after=\strut]
			\item Loose wires
			\item Incorrect communication protocol
			\item Incorrect parameters for serial packets (size, format, etc.)
		\end{itemize}
	\end{minipage} &
	\begin{minipage}[t]{\linewidth}
		\begin{itemize}[nosep, wide=0pt, leftmargin=*, after=\strut]
			\item Data can't be transferred between hardware and software
			\item Lost data
		\end{itemize}
	\end{minipage} &
	\begin{minipage}[t]{\linewidth}
		\begin{itemize}[nosep, wide=0pt, leftmargin=*, after=\strut]
			\item Check list of connected devices on device manager
			\item Visual inspection of wiring and circuitry
			\item Attach an error detection LED on the device
		\end{itemize}
	\end{minipage} &
	\begin{minipage}[t]{\linewidth}
		\begin{itemize}[nosep, wide=0pt, leftmargin=*, after=\strut]
			\item Make sure all necessary connections are made
			\item Reboot device
			\item Restart host software
		\end{itemize}
	\end{minipage}
	\tabularnewline\cline{1-6}		

\end{tabular}
%\vspace{3mm}

\section{Safety and Security Requirements}
\begin{itemize}[nosep, wide=0pt, leftmargin=*, after=\strut]
	\item All users are required to complete test prompt
	\item All data to be backed up each time the deivce connects to host software
	\item Only admin users will have access to device manager
\end{itemize}

\wss{Newly discovered requirements.  These should also be added to the SRS.  (A
rationale design process how and why to fake it.)}

\section{Roadmap}

\wss{Which safety requirements will be implemented as part of the capstone timeline?
Which requirements will be implemented in the future?}

\end{document}