\documentclass{article}

\usepackage{booktabs}
\usepackage{tabularx}
\usepackage{hyperref}
\usepackage{color, colortbl}
\usepackage{multirow,graphicx,array}
\usepackage{enumitem}
\definecolor{LightCyan}{rgb}{0.88,1,1}
\hypersetup{
    colorlinks=true,       % false: boxed links; true: colored links
    linkcolor=red,          % color of internal links (change box color with linkbordercolor)
    citecolor=green,        % color of links to bibliography
    filecolor=magenta,      % color of file links
    urlcolor=cyan           % color of external links
}

\title{Hazard Analysis\\\progname}

\author{\authname}

\date{}

%% Comments

\usepackage{color}

\newif\ifcomments\commentstrue %displays comments
%\newif\ifcomments\commentsfalse %so that comments do not display

\ifcomments
\newcommand{\authornote}[3]{\textcolor{#1}{[#3 ---#2]}}
\newcommand{\todo}[1]{\textcolor{red}{[TODO: #1]}}
\else
\newcommand{\authornote}[3]{}
\newcommand{\todo}[1]{}
\fi

\newcommand{\wss}[1]{\authornote{blue}{SS}{#1}} 
\newcommand{\plt}[1]{\authornote{magenta}{TPLT}{#1}} %For explanation of the template
\newcommand{\an}[1]{\authornote{cyan}{Author}{#1}}

%% Common Parts

\newcommand{\progname}{ProgName} % PUT YOUR PROGRAM NAME HERE
\newcommand{\authname}{Team \#, Team Name
\\ Student 1 name and macid
\\ Student 2 name and macid
\\ Student 3 name and macid
\\ Student 4 name and macid} % AUTHOR NAMES                  

\usepackage{hyperref}
    \hypersetup{colorlinks=true, linkcolor=blue, citecolor=blue, filecolor=blue,
                urlcolor=blue, unicode=false}
    \urlstyle{same}
                                


\begin{document}

\maketitle
\thispagestyle{empty}

~\newpage

\pagenumbering{roman}

\begin{table}[hp]
    \caption{Revision History} \label{TblRevisionHistory}
    \begin{tabularx}{\textwidth}{llX}
        \toprule
        \textbf{Date} & \textbf{Developer(s)} & \textbf{Change}        \\
        \midrule
        Date1         & Name(s)               & Description of changes \\
        Date2         & Name(s)               & Description of changes \\
        ...           & ...                   & ...                    \\
        \bottomrule
    \end{tabularx}
\end{table}

~\newpage

\tableofcontents

~\newpage

\pagenumbering{arabic}

\wss{You are free to modify this template.}

\section{Introduction}

\wss{You can include your definition of what a hazard is here.}

\section{Scope and Purpose of Hazard Analysis}

\section{System Boundaries and Components}

\section{Critical Assumptions}

\wss{These assumptions that are made about the software or system.  You should
    minimize the number of assumptions that remove potential hazards.  For instance,
    you could assume a part will never fail, but it is generally better to include
    this potential failure mode.}

\section{Failure Mode and Effect Analysis}
%% use this instead of \centerline
\begin{tabular}{|p{6em}|p{7em}|p{7em}|p{5em}|p{5em}|p{9em}|}
    \rowcolor{LightCyan}
    \textbf{Design Component} & \textbf{Failure Modes}          & \textbf{Causes of Failure}                                  & \textbf{Effects of Failure}                         & \textbf{Detection}                                        & \textbf{Recommended Action}\tabularnewline\hline
    %
                              & \begin{minipage}[t]{\linewidth}
                                    \begin{itemize}[nosep, wide=0pt, leftmargin=*, after=\strut]
            \item Heart rate not detected
            \item User's body temperature couldn't be recorded
            \item Surrounding pressure not being tracked
            \item No motion detected
        \end{itemize}
                                \end{minipage} &
    \begin{minipage}[t]{\linewidth}
        \begin{itemize}[nosep, wide=0pt, leftmargin=*, after=\strut]
            \item Device not worn properly
            \item Sensor breaks down, due to passing thresholds limits
            \item Device surface is dusty or contaminated
        \end{itemize}
    \end{minipage}
                              & EMA may not be triggered        & Software check to see if any sensor data is being collected &
    \begin{minipage}[t]{\linewidth}
        \begin{itemize}[nosep, wide=0pt, leftmargin=*, after=\strut]
            \item Ensure the device has been wrapped around properly for more accurate detection
            \item Reboot the system
            \item Let error handler try to solve the issue
        \end{itemize}
    \end{minipage}  \tabularnewline\cline{1-6}

    %
    \multirow{-16}{*}{\centering\rotatebox[origin=c]{90}{Sensor Array}}
                              & Device not starting             & \begin{minipage}[t]{\linewidth}
                                                                      \begin{itemize}[nosep, wide=0pt, leftmargin=*, after=\strut]
            \item Device is faulty
            \item Battery has died
            \item Device may have overheated
            \item Water may have damaged the battery
        \end{itemize}
                                                                  \end{minipage}                             & Device is not turned on, resulting in no monitoring & Battery dead indicator appears when turning on the device & \begin{minipage}[t]{\linewidth}
                                                                                                                                                                                                                                     \begin{itemize}[nosep, wide=0pt, leftmargin=*, after=\strut]
            \item Charge the device
            \item Change batteries if needed
        \end{itemize}
                                                                                                                                                                                                                                 \end{minipage}  \tabularnewline\cline{1-6}
    \multirow{-13}{*}{\centering\rotatebox[origin=c]{90}{Battery management system}}
\end{tabular}%\vspace{3mm}


\section{Safety and Security Requirements}

\section{Roadmap}
\subsection{Battery Requirements}
\textbf{BR1}: Fuses will be connected in the circuitry to shut down the device in case of voltages higher than threshold, as the batteries heat up.
\\
\textbf{BR2}: Adaptive charging can ensure that enough voltage is allowed to charge the battery. After it's fully charged, it can turn into open circuit to avoid damaging battery life.

\subsection{Sensory Array Requirements}
\textbf{SRA1}: In case a high-frequency emitting sensor is used, the limit will be set to ensure that the user isn't exposed to harmful radiation.
\\
\textbf{SRA2}: Secure network protocols will be used to ensure that the data can't be decrypted if intercepted by hackers.


\end{document}