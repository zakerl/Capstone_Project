\documentclass[12pt, titlepage]{article}

\usepackage{booktabs}
\usepackage{tabularx}
\usepackage{hyperref}
\hypersetup{
    colorlinks,
    citecolor=blue,
    filecolor=black,
    linkcolor=red,
    urlcolor=blue
}
\usepackage[round]{natbib}

%% Comments

\usepackage{color}

\newif\ifcomments\commentstrue %displays comments
%\newif\ifcomments\commentsfalse %so that comments do not display

\ifcomments
\newcommand{\authornote}[3]{\textcolor{#1}{[#3 ---#2]}}
\newcommand{\todo}[1]{\textcolor{red}{[TODO: #1]}}
\else
\newcommand{\authornote}[3]{}
\newcommand{\todo}[1]{}
\fi

\newcommand{\wss}[1]{\authornote{blue}{SS}{#1}} 
\newcommand{\plt}[1]{\authornote{magenta}{TPLT}{#1}} %For explanation of the template
\newcommand{\an}[1]{\authornote{cyan}{Author}{#1}}

%% Common Parts

\newcommand{\progname}{ProgName} % PUT YOUR PROGRAM NAME HERE
\newcommand{\authname}{Team \#, Team Name
\\ Student 1 name and macid
\\ Student 2 name and macid
\\ Student 3 name and macid
\\ Student 4 name and macid} % AUTHOR NAMES                  

\usepackage{hyperref}
    \hypersetup{colorlinks=true, linkcolor=blue, citecolor=blue, filecolor=blue,
                urlcolor=blue, unicode=false}
    \urlstyle{same}
                                


\begin{document}

\title{Project Title: System Verification and Validation Plan for \progname{}} 
\author{\authname}
\date{\today}
	
\maketitle

\pagenumbering{roman}

\section{Revision History}

\begin{tabularx}{\textwidth}{p{3cm}p{2cm}X}
\toprule {\bf Date} & {\bf Version} & {\bf Notes}\\
\midrule
November 2nd, 2022 & 1.0 & Initial Documentation\\
\bottomrule
\end{tabularx}

\newpage

\tableofcontents

\listoftables
\wss{Remove this section if it isn't needed}

\listoffigures
\wss{Remove this section if it isn't needed}

\newpage

\section{Symbols, Abbreviations and Acronyms}

\renewcommand{\arraystretch}{1.2}
\begin{tabular}{l l} 
  \toprule		
  \textbf{symbol} & \textbf{description}\\
  \midrule 
  T & Test\\
  \bottomrule
\end{tabular}\\

\wss{symbols, abbreviations or acronyms --- you can simply reference the SRS
  \citep{SRS} tables, if appropriate}

\wss{Remove this section if it isn't needed}

\newpage

\pagenumbering{arabic}

This document ... \wss{provide an introductory blurb and roadmap of the
  Verification and Validation plan}

\section{General Information}

\subsection{Summary}

Researchers at the School of Rehabilitation Sciences (SReS) at McMaster University are interested in performing Ecological Momentary Assessment (EMA) for victims of spinal disorders and back pain. EMA aims to study the thoughts, experiences, and behaviours of a participant's daily life by repeatedly collecting data in an individual's normal environment, at or close to the time they carry out that behaviour.\\

The type of EMA that the SReS is interested in is focused on analyzing the daily activities and symptoms of mostly-older adults with mobility and spinal issues. They wish to track that participant's walking activity as they go about their daily life, along with prompting participants with questions when relevant events occur. The answers to event-based prompts will be combined with activity monitoring data to form a better picture about the experience this participant has with their spinal and mobility issues.\\

The [name of device] will perform EMA analysis in a manner which even older participants can use, integrated into one package which gathers relevant data about a participant's activities and allows them to easily report what is currently going on and how they feel. They also are looking for a way to access EMA data in various ways. This includes graphical representations of the data which are meaningful to researchers, along with the raw data itself. This data could be activity data, symptoms reporting data, both types of data collated together, and so on. \\

This document is intended to describe the plan for verification and validation of the device. Here, verification and validation are both technical terms with very specific meanings.\\

Verification involves checking whether or not the specifications which were described in the planning phase of the project are implemented correctly by the final system designed. Validation involves checking whether or not the product actually fulfills the needs of the end user of the device. In the words of Barry Boehm, verification asks, "Are we building the product right?" and validation asks, "Are we building the right product?".\cite{pham_1999}

\subsection{Objectives}

The objectives aimed to be accomplished by this verification and validation are to:

\begin{itemize}
\item Validate if user requirements truly represent the goals of the stakeholders of [name of device].
\item Validate if the input provided by the operators of [name of device] meet the established rules and constraints.
\item Verify if the high-level design of the device correctly fulfills the specifications of the functional and non-functional requirements.
\item Verify if the components of the device (e.g. source code, database, physical construction, user interface, etc.) fulfill the specifications of the design. 
\item Discover any faults, failures, or malfunctions in the [name of device].
\end{itemize}

\subsection{Relevant Documentation}

Please refer to the following documentation for reference and more information:
\wss{Reference relevant documentation.  This will definitely include your SRS
  and your other project documents (design documents, like MG, MIS, etc).  You
  can include these even before they are written, since by the time the project
  is done, they will be written.}
\begin{itemize}
\item Development Plan: \citet{Development_Plan}
\item Problem Statement \& Goals: \citet{Prob_n_Goals}
\item SRS: \citet{SRS}
\item Hazard Analysis: \citet{Hazard_Analysis}
\item VnV Plan: \citet{VnV_Plan}
\item Reflection: \citet{Reflection}
\item User Guide: \citet{User_Guide}
\end{itemize}

\section{Plan}

\wss{Introduce this section.   You can provide a roadmap of the sections to
  come.}

\subsection{Verification and Validation Team}

\wss{Your teammates.  Maybe your supervisor.
  You shoud do more than list names.  You should say what each person's role is
  for the project's verification.  A table is a good way to summarize this information.}

\subsection{SRS Verification Plan}

\wss{List any approaches you intend to use for SRS verification.  This may include
  ad hoc feedback from reviewers, like your classmates, or you may plan for 
  something more rigorous/systematic.}

\wss{Maybe create an SRS checklist?}

\subsection{Design Verification Plan}

\wss{Plans for design verification}

\wss{The review will include reviews by your classmates}

\wss{Create a checklists?}

\subsection{Verification and Validation Plan Verification Plan}

\wss{The verification and validation plan is an artifact that should also be verified.}

\wss{The review will include reviews by your classmates}

\wss{Create a checklists?}

\subsection{Implementation Verification Plan}

\wss{You should at least point to the tests listed in this document and the unit
  testing plan.}

\wss{In this section you would also give any details of any plans for static verification of
  the implementation.  Potential techniques include code walkthroughs, code
  inspection, static analyzers, etc.}

\subsection{Automated Testing and Verification Tools}

\wss{What tools are you using for automated testing.  Likely a unit testing
  framework and maybe a profiling tool, like ValGrind.  Other possible tools
  include a static analyzer, make, continuous integration tools, test coverage
  tools, etc.  Explain your plans for summarizing code coverage metrics.
  Linters are another important class of tools.  For the programming language
  you select, you should look at the available linters.  There may also be tools
  that verify that coding standards have been respected, like flake9 for
  Python.}

\wss{If you have already done this in the development plan, you can point to
that document.}

\wss{The details of this section will likely evolve as you get closer to the
  implementation.}

\subsection{Software Validation Plan}

\wss{If there is any external data that can be used for validation, you should
  point to it here.  If there are no plans for validation, you should state that
  here.}

\wss{You might want to use review sessions with the stakeholder to check that
the requirements document captures the right requirements.  Maybe task based
inspection?}

\wss{This section might reference back to the SRS verification section.}

\section{System Test Description}
	
\subsection{Tests for Functional Requirements}

\wss{Subsets of the tests may be in related, so this section is divided into
  different areas.  If there are no identifiable subsets for the tests, this
  level of document structure can be removed.}

\wss{Include a blurb here to explain why the subsections below
  cover the requirements.  References to the SRS would be good here.}

\subsubsection{Area of Testing1}

\wss{It would be nice to have a blurb here to explain why the subsections below
  cover the requirements.  References to the SRS would be good here.  If a section
  covers tests for input constraints, you should reference the data constraints
  table in the SRS.}
		
\paragraph{Title for Test}

\begin{enumerate}

\item{test-id1\\}

Control: Manual versus Automatic
					
Initial State: 
					
Input: 
					
Output: \wss{The expected result for the given inputs}

Test Case Derivation: \wss{Justify the expected value given in the Output field}
					
How test will be performed: 
					
\item{test-id2\\}

Control: Manual versus Automatic
					
Initial State: 
					
Input: 
					
Output: \wss{The expected result for the given inputs}

Test Case Derivation: \wss{Justify the expected value given in the Output field}

How test will be performed: 

\end{enumerate}

\subsubsection{Area of Testing2}

...

\subsection{Tests for Nonfunctional Requirements}

\wss{The nonfunctional requirements for accuracy will likely just reference the
  appropriate functional tests from above.  The test cases should mention
  reporting the relative error for these tests.  Not all projects will
  necessarily have nonfunctional requirements related to accuracy}

\wss{Tests related to usability could include conducting a usability test and
  survey.  The survey will be in the Appendix.}

\wss{Static tests, review, inspections, and walkthroughs, will not follow the
format for the tests given below.}

\subsubsection{Area of Testing1}
		
\paragraph{Title for Test}

\begin{enumerate}

\item{test-id1\\}

Type: Functional, Dynamic, Manual, Static etc.
					
Initial State: 
					
Input/Condition: 
					
Output/Result: 
					
How test will be performed: 
					
\item{test-id2\\}

Type: Functional, Dynamic, Manual, Static etc.
					
Initial State: 
					
Input: 
					
Output: 
					
How test will be performed: 

\end{enumerate}

\subsubsection{Area of Testing2}

...

\subsection{Traceability Between Test Cases and Requirements}

\wss{Provide a table that shows which test cases are supporting which
  requirements.}

\section{Unit Test Description}

\wss{Reference your MIS (detailed design document) and explain your overall
  philosophy for test case selection.}  
\wss{This section should not be filled in until after the MIS (detailed design
  document) has been completed.}

\subsection{Unit Testing Scope}

\wss{What modules are outside of the scope.  If there are modules that are
  developed by someone else, then you would say here if you aren't planning on
  verifying them.  There may also be modules that are part of your software, but
  have a lower priority for verification than others.  If this is the case,
  explain your rationale for the ranking of module importance.}

\subsection{Tests for Functional Requirements}

\wss{Most of the verification will be through automated unit testing.  If
  appropriate specific modules can be verified by a non-testing based
  technique.  That can also be documented in this section.}

\subsubsection{Module 1}

\wss{Include a blurb here to explain why the subsections below cover the module.
  References to the MIS would be good.  You will want tests from a black box
  perspective and from a white box perspective.  Explain to the reader how the
  tests were selected.}

\begin{enumerate}

\item{test-id1\\}

Type: \wss{Functional, Dynamic, Manual, Automatic, Static etc. Most will
  be automatic}
					
Initial State: 
					
Input: 
					
Output: \wss{The expected result for the given inputs}

Test Case Derivation: \wss{Justify the expected value given in the Output field}

How test will be performed: 
					
\item{test-id2\\}

Type: \wss{Functional, Dynamic, Manual, Automatic, Static etc. Most will
  be automatic}
					
Initial State: 
					
Input: 
					
Output: \wss{The expected result for the given inputs}

Test Case Derivation: \wss{Justify the expected value given in the Output field}

How test will be performed: 

\item{...\\}
    
\end{enumerate}

\subsubsection{Module 2}

...

\subsection{Tests for Nonfunctional Requirements}

\wss{If there is a module that needs to be independently assessed for
  performance, those test cases can go here.  In some projects, planning for
  nonfunctional tests of units will not be that relevant.}

\wss{These tests may involve collecting performance data from previously
  mentioned functional tests.}

\subsubsection{Module ?}
		
\begin{enumerate}

\item{test-id1\\}

Type: \wss{Functional, Dynamic, Manual, Automatic, Static etc. Most will
  be automatic}
					
Initial State: 
					
Input/Condition: 
					
Output/Result: 
					
How test will be performed: 
					
\item{test-id2\\}

Type: Functional, Dynamic, Manual, Static etc.
					
Initial State: 
					
Input: 
					
Output: 
					
How test will be performed: 

\end{enumerate}

\subsubsection{Module ?}

...

\subsection{Traceability Between Test Cases and Modules}

\wss{Provide evidence that all of the modules have been considered.}
				
\bibliographystyle{plainnat}

\bibliography{../../refs/References}

\newpage

\section{Appendix}

This is where you can place additional information.

\subsection{Symbolic Parameters}

The definition of the test cases will call for SYMBOLIC\_CONSTANTS.
Their values are defined in this section for easy maintenance.

\subsection{Usability Survey Questions?}

\wss{This is a section that would be appropriate for some projects.}

\newpage{}
\section*{Appendix --- Reflection}

The information in this section will be used to evaluate the team members on the
graduate attribute of Lifelong Learning.  Please answer the following questions:

\newpage{}
\section*{Appendix --- Reflection}

The information in this section will be used to evaluate the team members on the
graduate attribute of Lifelong Learning.  Please answer the following questions:

\begin{enumerate}
  \item What knowledge and skills will the team collectively need to acquire to
  successfully complete the verification and validation of your project?
  Examples of possible knowledge and skills include dynamic testing knowledge,
  static testing knowledge, specific tool usage etc.  You should look to
  identify at least one item for each team member.
  \item For each of the knowledge areas and skills identified in the previous
  question, what are at least two approaches to acquiring the knowledge or
  mastering the skill?  Of the identified approaches, which will each team
  member pursue, and why did they make this choice?
\end{enumerate}

\end{document}