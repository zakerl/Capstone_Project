\documentclass[12pt, titlepage]{article}

\usepackage{booktabs}
\usepackage{tabularx}
\usepackage{hyperref}
\usepackage{float}
\usepackage{zref-xr,zref-user}

\hypersetup{
    colorlinks,
    citecolor=blue,
    filecolor=black,
    linkcolor=red,
    urlcolor=blue
}
\usepackage[round]{natbib}

\input{../Comments}
%% Common Parts

\newcommand{\progname}{Mechatronics Engineering} % PUT YOUR PROGRAM NAME HERE
\newcommand{\authname}{Team \#1, Back End Developers
\\ Jessica Bae
\\ Oliver Foote
\\ Jonathan Hai
\\ Anish Rangarajan
\\ Nish Shah
\\ Labeeb Zaker} % AUTHOR NAMES                  

\usepackage{hyperref}
    \hypersetup{colorlinks=true, linkcolor=blue, citecolor=blue, filecolor=blue,
                urlcolor=blue, unicode=false}
    \urlstyle{same}

\begin{document}

\title{Project Title: System Verification and Validation Plan for \progname{}} 
\author{\authname}
\date{\today}
	
\maketitle

\pagenumbering{roman}

\section{Revision History}

\begin{tabularx}{\textwidth}{p{3cm}p{2cm}X}
\toprule {\bf Date} & {\bf Version} & {\bf Notes}\\
\midrule
Date 1 & 1.0 & Notes\\
Date 2 & 1.1 & Notes\\
\bottomrule
\end{tabularx}

\newpage

\tableofcontents

\listoftables
\wss{Remove this section if it isn't needed}

\listoffigures
\wss{Remove this section if it isn't needed}

\newpage

\section{Symbols, Abbreviations and Acronyms}
%The test plan is \href{file:../SRS/SRS}{here}.
\renewcommand{\arraystretch}{1.2}
\begin{table}[H]
	\begin{tabular}{l l} 
		  \toprule		
			  \textbf{Symbol} & \textbf{Description}\\
			  \midrule 
			  DT 						& Duration Test\\
			  MTT 					& Minor Tracking Test\\
			  PT 						& Prompt Test\\
			  TT 						& Threshold Test\\
			  DST 					& Data Storage Test\\
			  DXT 					& Data Extraction Test\\
			  HC\#					& Hardware Constraint \#\\ 
		  \bottomrule
	\end{tabular}\\\\
\caption{\label{Syb}Table of Symbols}
\end{table}

\wss{symbols, abbreviations or acronyms --- you can simply reference the SRS
  \citep{SRS} tables, if appropriate}

\wss{Remove this section if it isn't needed}

\newpage

\pagenumbering{arabic}

This document ... \wss{provide an introductory blurb and roadmap of the
  Verification and Validation plan}

\section{General Information}

\subsection{Summary}

\wss{Say what software is being tested.  Give its name and a brief overview of
  its general functions.}

\subsection{Objectives}

\wss{State what is intended to be accomplished.  The objective will be around
  the qualities that are most important for your project.  You might have
  something like: ``build confidence in the software correctness,''
  ``demonstrate adequate usability.'' etc.  You won't list all of the qualities,
  just those that are most important.}

\subsection{Relevant Documentation}

\wss{Reference relevant documentation.  This will definitely include your SRS
  and your other project documents (design documents, like MG, MIS, etc).  You
  can include these even before they are written, since by the time the project
  is done, they will be written.}

\citet{SRS}

\section{Plan}

\wss{Introduce this section.   You can provide a roadmap of the sections to
  come.}

\subsection{Verification and Validation Team}

\wss{Your teammates.  Maybe your supervisor.
  You shoud do more than list names.  You should say what each person's role is
  for the project's verification.  A table is a good way to summarize this information.}

\subsection{SRS Verification Plan}

\wss{List any approaches you intend to use for SRS verification.  This may include
  ad hoc feedback from reviewers, like your classmates, or you may plan for 
  something more rigorous/systematic.}

\wss{Maybe create an SRS checklist?}

\subsection{Design Verification Plan}

\wss{Plans for design verification}

\wss{The review will include reviews by your classmates}

\wss{Create a checklists?}

\subsection{Verification and Validation Plan Verification Plan}

\wss{The verification and validation plan is an artifact that should also be verified.}

\wss{The review will include reviews by your classmates}

\wss{Create a checklists?}

\subsection{Implementation Verification Plan}

\wss{You should at least point to the tests listed in this document and the unit
  testing plan.}

\wss{In this section you would also give any details of any plans for static verification of
  the implementation.  Potential techniques include code walkthroughs, code
  inspection, static analyzers, etc.}

\subsection{Automated Testing and Verification Tools}

\wss{What tools are you using for automated testing.  Likely a unit testing
  framework and maybe a profiling tool, like ValGrind.  Other possible tools
  include a static analyzer, make, continuous integration tools, test coverage
  tools, etc.  Explain your plans for summarizing code coverage metrics.
  Linters are another important class of tools.  For the programming language
  you select, you should look at the available linters.  There may also be tools
  that verify that coding standards have been respected, like flake9 for
  Python.}

\wss{If you have already done this in the development plan, you can point to
that document.}

\wss{The details of this section will likely evolve as you get closer to the
  implementation.}

\subsection{Software Validation Plan}

\wss{If there is any external data that can be used for validation, you should
  point to it here.  If there are no plans for validation, you should state that
  here.}

\wss{You might want to use review sessions with the stakeholder to check that
the requirements document captures the right requirements.  Maybe task based
inspection?}

\wss{This section might reference back to the SRS verification section.}

\section{System Test Description}
	
\subsection{Tests for Functional Requirements}

\wss{Subsets of the tests may be in related, so this section is divided into
  different areas.  If there are no identifiable subsets for the tests, this
  level of document structure can be removed.}

\wss{Include a blurb here to explain why the subsections below
  cover the requirements.  References to the SRS would be good here.}

\wss{It would be nice to have a blurb here to explain why the subsections below
  cover the requirements.  References to the SRS would be good here.  If a section
  covers tests for input constraints, you should reference the data constraints
  table in the SRS.}

%%%%%%%%%%%%%%%%%%%%%%%%%%%%%%%%%%%%%%%%%%%%%%%%%%%%%%%%%%%%%%%%%%%%%%%%%%%%%%%%%%%%%%%%%%%%%%%%%%%%%%%%
\subsubsection{Device should stay on during the monitoring period}

The following tests will check the whether the device will maintain an "ON" state throughout the duration of the monitoring period. The primary tests will involve different monitoring periods with valid inputs, invalid inputs, dates from the past or too far into the future beyond what the battery life can sustain (HC2).
		
\paragraph{Duration Test}

\begin{enumerate}

	\item{\textbf{DT\_1: Regular Inputs}\\}\label{DT1}
		
		Control: Manual 
							
		Initial State: Device waits for the monitoring period to be set up in the configuration of the device.
							
		Input: Monitoring Period ["Date","Time"]  : ["03-11-2022", "05:30:PM"].
		
		Output: Device turns off after Monitoring Period.
							
		How test will be performed: The test is performed by passing in the Monitoring Period and ensuring that the device maintains power throughout this period.
					
	\item{\textbf{DT\_2: Invalid Inputs}\\}\label{DT2}
	
		Control: Manual 
							
		Initial State: Device waits for the monitoring period to be set up in the configuration of the device.
							
		Input: Monitoring Period ["Date","Time"] : ["3rd November 2022", "Five Thirty PM"].
							
		Output: Device returns an error code to the Error Handler and asserts the Invalid Data Error (BED\_ERR\_INVALID\_DATA).
		
		How test will be performed: The test is performed by passing an invalid input and ensuring the appropriate error code is returned.

	\item{\textbf{DT\_3: Earlier Date}\\}\label{DT3}
		
		Control: Manual 
							
		Initial State: Device waits for the monitoring period to be set up in the configuration of the device.
							
		Input: Monitoring Period ["Date","Time"] : ["01-1-1999", "05:30:PM"].
							
		Output: Device returns an error code to the Error Handler and asserts the Invalid Data Error (BED\_ERR\_INVALID\_DATA).
		
		How test will be performed: The test is performed by passing an old date (prior to current date).

	\item{\textbf{DT\_4: Date beyond capabilities}\\}\label{DT4}
		
		Control: Manual 
							
		Initial State: Device waits for the monitoring period to be set up in the configuration of the device.
							
		Input: Monitoring Period ["Date","Time"] : ["9-12-2300", "05:30:PM"].
							
		Output: Device returns an error code to the Error Handler and asserts the Invalid Data Error (BED\_ERR\_INVALID\_DATA).

		How test will be performed: The test is performed by passing a date greater than what the battery life of the device can support.

\end{enumerate}
%%%%%%%%%%%%%%%%%%%%%%%%%%%%%%%%%%%%%%%%%%%%%%%%%%%%%%%%%%%%%%%%%%%%%%%%%%%%%%%%%%%%%%%%%%%%%%%%%%%%%%%%

\subsubsection{Device should track Minor Movements}
The following tests are run to ensure that the device is able to track activities to a resolution deemed sufficient for general activity tracking. Tests will involve varying the rate at which the device is moved,rotated oriented etc. and checking if the status of the Sensors is valid.

\paragraph{Minor Tracking Test}
\begin{enumerate}
	\item{\textbf{MTT\_1: Regular Movement} \\}\label{MTT1}
	
		Control: Manual 
							
		Initial State: Device is worn with all systems working.
							
		Input: Wearer performs activities at a regular/normal pace.
		
		Output: Status returned by the Sensor Array is no error (BED\_ERR\_NONE).

		How test will be performed: The test is performed by strapping the device onto a test volunteer who will perform the tracked activities at a normal/regular pace. This is done to ensure that the device can 				work under normal scenarios.\\


	\item{\textbf{MTT\_2: Slow Movement} \\}\label{MTT2}
	
		Control: Manual 
							
		Initial State: Device is worn with all systems working.
							
		Input: Wearer performs activities at a very slow pace.
		
		Output: Status returned by the Sensor Array is no error (BED\_ERR\_NONE).
		
		How test will be performed: The test is performed by strapping the device onto a test volunteer who will perform the tracked activities at a very slow pace. This is done to ensure that the device can 					work under scenarios in which users have limited mobility.\\


\end{enumerate}
%%%%%%%%%%%%%%%%%%%%%%%%%%%%%%%%%%%%%%%%%%%%%%%%%%%%%%%%%%%%%%%%%%%%%%%%%%%%%%%%%%%%%%%%%%%%%%%%%%%%%%%%
\subsubsection{Device prompts user when activity is detected}

The following tests will be done to ensure that the proper info is prompted to the user when activities are detected. Tests involve generating the test prompt, generating prompts for different activities.
		
\paragraph{Prompt Tests}
\begin{enumerate}
	\item{\textbf{PT\_1: Test Prompt} \\}\label{PT1}
	
		Control: Manual 
							
		Initial State: Device has just been reconfigured.
							
		Input: Device is turned on for the first time.
		
		Output: Test Prompt is displayed.

		How test will be performed: The test is performed by turning on the device for the first time after reconfiguration. Upon turning on the device, the user should receive a test prompt that will confirm that the 				prompting system is working correctly.\\
		Test Prompt: Is this Device on?\\
		Possible Answers: Yes or No
		
	\item{\textbf{PT\_2: Activity Prompt} \\}\label{PT2}
	
		Control: Manual 
							
		Initial State: Device is in the idle state.
							
		Input: Activity has been detected.
		
		Output: Specific activity prompt is displayed.
							
		How test will be performed: The test is done by having a test volunteer perform one of the activities that are registered. This should result in a prompt for the volunteer that is generated based on the 					specific activity performed.\\\\
		Tracked Activity: Participant slows down or comes to a stop.\\
		Activity Prompt: Are you in pain?\\
		Possible Answers: Yes or No
\end{enumerate}
%%%%%%%%%%%%%%%%%%%%%%%%%%%%%%%%%%%%%%%%%%%%%%%%%%%%%%%%%%%%%%%%%%%%%%%%%%%%%%%%%%%%%%%%%%%%%%%%%%%%%%%%

\subsubsection{Customizable Thresholds}

The following tests will be done to ensure that the proper info is prompted to the user when activities are detected. Tests involve generating the test prompt, generating prompts for different activities.
		
\paragraph{Threshold Tests}
\begin{enumerate}
	\item{\textbf{TT\_1: Regular Inputs} \\}\label{TT1}
	
		Control: Manual 
							
		Initial State: Device is in Configuration mode.
							
		Input: Regular values for thresholds within limits.
		
		Output: Config File generated successfully.
		
		How test will be performed: The test is performed by setting the device to configuration mode and then setting valid values for the thresholds.\\
		eg: \\
		Speed Threshold:\\
		Limits: 0m/s \textless Threshold \textless 5m/s

	\item{\textbf{TT\_2: Below lower limit} \\}\label{TT2}
	
		Control: Manual 
							
		Initial State: Device is in Configuration mode.
							
		Input: Values for thresholds below lower limits.
		
		Output: Config File generates a BED\_ERR\_OUT\_OF\_BOUNDS.

		How test will be performed: The test is performed by setting the device to configuration mode and then setting values for the thresholds above allowable limits.

	\item{\textbf{TT\_3: Above Upper limit }\\}\label{TT3}
	
		Control: Manual 
							
		Initial State: Device is in Configuration mode.
							
		Input: Values for thresholds above upper limits.
		
		Output: Config File generates a BED\_ERR\_OUT\_OF\_BOUNDS.
		
		How test will be performed: The test is performed by setting the device to configuration mode and then setting values for the thresholds below allowable limits.

	\item{\textbf{TT\_4: Invalid Value} \\}\label{TT4}
	
		Control: Manual 
							
		Initial State: Device is in Configuration mode.
							
		Input: Invalid values for thresholds.
		
		Output: Config File generates a BED\_ERR\_INVALID\_DATA.
		
		How test will be performed: The test is performed by setting the device to configuration mode and then setting invalid values for the thresholds.

	\item{\textbf{TT\_5: No Value} \\}\label{TT5}
	
		Control: Manual 
							
		Initial State: Device is in Configuration mode.
							
		Input: No values for thresholds.
		
		Output: Config File generates a BED\_ERR\_INVALID\_DATA.

		How test will be performed: The test is performed by setting the device to configuration mode and then setting no values for the thresholds.
\end{enumerate}
%%%%%%%%%%%%%%%%%%%%%%%%%%%%%%%%%%%%%%%%%%%%%%%%%%%%%%%%%%%%%%%%%%%%%%%%%%%%%%%%%%%%%%%%%%%%%%%%%%%%%%%%

\subsubsection{Data Storage}

The following tests will be done to ensure that data is stored when appropriate. Tests include checking when storage buffer is full, when prompts are generated and when sensor data needs to be logged. 
\begin{enumerate}
	\item{\textbf{DST\_1: Storage Buffer Full} \\}\label{DST1}
	
		Control: Manual 
							
		Initial State: Device is in an idle state.
							
		Input: Activity detected causing prompt to be generated (Internal storage is full).
		
		Output: Data Storage system generates a BED\_ERR\_MEMORY\_FULL
		
		How test will be performed: The test is performed by first loading the internal memory buffer with garbage values so that it is nearly/completely full. Then a registered activity is triggered generating a 					prompt. Once this is answered, the system will not have enough memory to store the new values thus resulting in an error.

	\item{\textbf{DST\_2: Prompt Generated} \\}\label{DST2}
	
		Control: Manual 
							
		Initial State: Device is in an idle state.
							
		Input: Activity detected causing prompt to be generated.
		
		Output: Prompt Response is saved into the internal memory.
		
		How test will be performed: The test is performed by performing a registered activity and ensuring that the prompt generated is answered and its result is stored in the internal storage buffer.\\
		eg: Registered Activity: Participant slows down or comes to a stop for an extended period of time\\
		Prompt Generated: \\
		Are you in pain?\\
		Possible Answers: Yes or No\\
		\\
		Prompt Generated: \\
		Do you need assistance?\\
		Possible Answers: Yes or No\\

	\item{\textbf{DST\_3: Sensor Storage} \\}\label{DST3}
	
		Control: Manual 
							
		Initial State: Device is in an idle state.
							
		Input: Activity detected causing prompt to be generated.
		
		Output: Specific sensor values that triggered a prompt are stored in the internal memory.
							
		How test will be performed: The test is performed by performing a registered activity and ensuring that the sensor values that caused the prompt are stored in the internal storage buffer.
\end{enumerate}
%%%%%%%%%%%%%%%%%%%%%%%%%%%%%%%%%%%%%%%%%%%%%%%%%%%%%%%%%%%%%%%%%%%%%%%%%%%%%%%%%%%%%%%%%%%%%%%%%%%%%%%%
\subsubsection{Data Extraction}

The following tests will be done to ensure that can be extracted and presented in a graphical manner deemed acceptable for the purpose of EMA analysis.
\begin{enumerate}
	\item{\textbf{DXT\_1: Extracting Data} \\}\label{DXT1}
	
		Control: Manual 
							
		Initial State: Device is connected to the device manager.
							
		Input: Command that tells the device manager to extract all data from the internal memory.
		
		Output: Extracted data is sent to the Host Software where it is converted to a presentable form.

		How test will be performed: The test is performed by first running the device as intended and waiting for a small monitoring period to finish. After this the device is connected to the Host Software with the 				help of the Device Manager Driver. Once connected, the user can start the extraction process and should be able to see all the relevant data in a presentable manner.
		
	\item{\textbf{DXT\_2: Extracting No Data} \\}\label{DXT2}
	
		Control: Manual 
							
		Initial State: Device is connected to the device manager.
							
		Input: Command that tells the device manager to extract all data from the internal memory.
		
		Output: Device Manager returns a BED\_ERR\_EMPTY\_DATA error

		How test will be performed: The test is performed by first deleting all the contents of the internal memory prior to connection with the Host Software. Once connected and the extraction process begins, the 				system should return an error due to no data being present to extract.

	\item{\textbf{DXT\_3: Extracting Corrupted Data} \\}\label{DXT3}
	
		Control: Manual 
							
		Initial State: Device is connected to the device manager.
							
		Input: Command that tells the device manager to extract all data from the internal memory.
		
		Output: Device Manager returns a BED\_ERR\_INVALID\_DATA error

		How test will be performed: The test is performed by first deleting all the contents of the internal memory and filling it with garbage values prior to connection with the Host Software. Once connected and 				the extraction process begins, the system should return an error due to corrupted data being present to extract.
\end{enumerate}
%%%%%%%%%%%%%%%%%%%%%%%%%%%%%%%%%%%%%%%%%%%%%%%%%%%%%%%%%%%%%%%%%%%%%%%%%%%%%%%%%%%%%%%%%%%%%%%%%%%%%%%%
\subsection{Tests for Nonfunctional Requirements}

\wss{The nonfunctional requirements for accuracy will likely just reference the
  appropriate functional tests from above.  The test cases should mention
  reporting the relative error for these tests.  Not all projects will
  necessarily have nonfunctional requirements related to accuracy}

\wss{Tests related to usability could include conducting a usability test and
  survey.  The survey will be in the Appendix.}

\wss{Static tests, review, inspections, and walkthroughs, will not follow the
format for the tests given below.}

\subsubsection{Area of Testing1}
		
\paragraph{Title for Test}

\begin{enumerate}

\item{test-id1\\}

Type: Functional, Dynamic, Manual, Static etc.
					
Initial State: 
					
Input/Condition: 
					
Output/Result: 
					
How test will be performed: 
					
\item{test-id2\\}

Type: Functional, Dynamic, Manual, Static etc.
					
Initial State: 
					
Input: 
					
Output: 
					
How test will be performed: 

\end{enumerate}

\subsubsection{Area of Testing2}

...

\subsection{Traceability Between Test Cases and Requirements}

\wss{Provide a table that shows which test cases are supporting which
  requirements.}

\section{Unit Test Description}

\wss{Reference your MIS (detailed design document) and explain your overall
  philosophy for test case selection.}  
\wss{This section should not be filled in until after the MIS (detailed design
  document) has been completed.}

\subsection{Unit Testing Scope}

\wss{What modules are outside of the scope.  If there are modules that are
  developed by someone else, then you would say here if you aren't planning on
  verifying them.  There may also be modules that are part of your software, but
  have a lower priority for verification than others.  If this is the case,
  explain your rationale for the ranking of module importance.}

\subsection{Tests for Functional Requirements}

\wss{Most of the verification will be through automated unit testing.  If
  appropriate specific modules can be verified by a non-testing based
  technique.  That can also be documented in this section.}

\subsubsection{Module 1}

\wss{Include a blurb here to explain why the subsections below cover the module.
  References to the MIS would be good.  You will want tests from a black box
  perspective and from a white box perspective.  Explain to the reader how the
  tests were selected.}

\begin{enumerate}

\item{test-id1\\}

Type: \wss{Functional, Dynamic, Manual, Automatic, Static etc. Most will
  be automatic}
					
Initial State: 
					
Input: 
					
Output: \wss{The expected result for the given inputs}

Test Case Derivation: \wss{Justify the expected value given in the Output field}

How test will be performed: 
					
\item{test-id2\\}

Type: \wss{Functional, Dynamic, Manual, Automatic, Static etc. Most will
  be automatic}
					
Initial State: 
					
Input: 
					
Output: \wss{The expected result for the given inputs}

Test Case Derivation: \wss{Justify the expected value given in the Output field}

How test will be performed: 

\item{...\\}
    
\end{enumerate}

\subsubsection{Module 2}

...

\subsection{Tests for Nonfunctional Requirements}

\wss{If there is a module that needs to be independently assessed for
  performance, those test cases can go here.  In some projects, planning for
  nonfunctional tests of units will not be that relevant.}

\wss{These tests may involve collecting performance data from previously
  mentioned functional tests.}

\subsubsection{Module ?}
		
\begin{enumerate}

\item{test-id1\\}

Type: \wss{Functional, Dynamic, Manual, Automatic, Static etc. Most will
  be automatic}
					
Initial State: 
					
Input/Condition: 
					
Output/Result: 
					
How test will be performed: 
					
\item{test-id2\\}

Type: Functional, Dynamic, Manual, Static etc.
					
Initial State: 
					
Input: 
					
Output: 
					
How test will be performed: 

\end{enumerate}

\subsubsection{Module ?}

...

\subsection{Traceability Between Test Cases and Modules}

\wss{Provide evidence that all of the modules have been considered.}
				
%\bibliographystyle{plainnat}

%\bibliography{../../refs/References}

\newpage

\section{Appendix}

This is where you can place additional information.

\subsection{Symbolic Parameters}

The definition of the test cases will call for SYMBOLIC\_CONSTANTS.
Their values are defined in this section for easy maintenance.
\begin{flushleft} 
	\begin{table}[H]
		\begin{tabular}{l l} 
			  \toprule		
				  \textbf{Error} & \textbf{Description}\\
				  \midrule 
				  BED\_ERR\_NONE 					& Represents no errors\\
				  BED\_ERR\_INVALID\_DATA			& Represents invalid data being used/stored\\
				  BED\_ERR\_INVALID\_DATA\_SIZE 		& Represents insufficient size for data storage\\
				  BED\_ERR\_OUT\_OF\_BOUNDS 		& Represents value of data past allowable limits\\
				  BED\_ERR\_MEMORY\_FULL 			& Represents full internal memory buffer\\
		
			  \bottomrule
		\end{tabular}\\
		\caption{\label{Err}Table of Errors}

	\end{table}
\end{flushleft} 
\subsection{Usability Survey Questions?}

\wss{This is a section that would be appropriate for some projects.}
\newpage{}
\section*{Appendix --- Reflection}

The information in this section will be used to evaluate the team members on the
graduate attribute of Lifelong Learning.  Please answer the following questions:

	\begin{enumerate}
	  \item What knowledge and skills will the team collectively need to acquire to
	  successfully complete the verification and validation of your project?
	  Examples of possible knowledge and skills include dynamic testing knowledge,
	  static testing knowledge, specific tool usage etc.  You should look to
	  identify at least one item for each team member.
	  \item For each of the knowledge areas and skills identified in the previous
	  question, what are at least two approaches to acquiring the knowledge or
	  mastering the skill?  Of the identified approaches, which will each team
	  member pursue, and why did they make this choice?
	\end{enumerate}

\end{document}